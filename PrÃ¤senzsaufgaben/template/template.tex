\documentclass[12pt,german]{article}
\usepackage[latin1]{inputenc}
\usepackage{textcomp}
\usepackage{epsf}
\usepackage{german}
\usepackage{float}
\usepackage{hyperref}
\usepackage{amsmath,amsfonts,amssymb,amsxtra}  % z.B. f�r \text im math-modus
\usepackage{pstricks,pst-node,pst-text,pst-3d}
\usepackage{graphicx}
\pagestyle{empty}
\usepackage{tikz}
\evensidemargin0cm
\oddsidemargin0cm
\textwidth16cm
\textheight24cm
\topmargin-2cm
\parindent 0pt
\newtheorem{aufgabe}{Aufgabe}

\begin{document}
\begin{flushright} Stanke/Hellmuth/Becker \end{flushright}
\vspace{-5mm}
\rule{\linewidth}{0.6mm}
\begin{center}{ \Large Datenstrukturen und Effiziente Algorithmen}\\
{\large Wintersemester 2019}\vspace{-3mm}
\end{center}
\rule{\linewidth}{0.6mm}
\begin{flushleft}
{\bf \large Pr�senzaufgaben 1}
\end{flushleft}

\vspace*{3mm}

\bigskip
{\bf Aufgabe 1.} (Keywords)\\
Aufgabenbeschreibung

\medskip
\textit{Kursiver Absatz}
\medskip

Weitere Beschreibung

\end{document}
