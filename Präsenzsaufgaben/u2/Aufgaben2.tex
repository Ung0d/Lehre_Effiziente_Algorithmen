\documentclass[12pt,german]{article}
\usepackage[latin1]{inputenc}
\usepackage{textcomp}
\usepackage{epsf}
\usepackage{german}
\usepackage{float}
\usepackage{hyperref}
\hypersetup{
    colorlinks=true,
    linkcolor=blue,
    filecolor=magenta,      
    urlcolor=cyan,
}
\usepackage{amsmath,amsfonts,amssymb,amsxtra}  % z.B. f�r \text im math-modus
\usepackage{pstricks,pst-node,pst-text,pst-3d}
\usepackage{graphicx}
\pagestyle{empty}
\usepackage{tikz}
\evensidemargin0cm
\oddsidemargin0cm
\textwidth16cm
\textheight24cm
\topmargin-2cm
\parindent 0pt
\newtheorem{aufgabe}{Aufgabe}

\begin{document}
\begin{flushright} Becker \end{flushright}
\vspace{-5mm}
\rule{\linewidth}{0.6mm}
\begin{center}{ \Large Datenstrukturen und Effiziente Algorithmen}\\
{\large Wintersemester 2019}\vspace{-3mm}
\end{center}
\rule{\linewidth}{0.6mm}
\begin{flushleft}
{\bf \large Pr�senzaufgaben 2}
\end{flushleft}

\vspace*{3mm}

{\bf Aufgabe 1.} (Fluss-Skalarprodukt)\\

Sei $f$ ein Fluss in einem Netzwerk und $\alpha \in \mathbb{R}$ Das Fluss-Skalarprodukt $\alpha f$ ist eine Funktion $V \times V \rightarrow \mathbb{R}$ definiert durch
$$ (\alpha f)(u, v) = \alpha \cdot f(u,v)$$
Zeige, dass falls $f_1$ und $f_2$ Fl�sse sind, so auch $\alpha f_1 + (1-\alpha) f_2$ f�r alle $\alpha$ zwischen 0 und 1.

\bigskip

{\bf Aufgabe 2.} (Knotenkapazit�ten)\\

Nehme an zus�tzlich zu Kantenkapazit�ten hat ein Flussnetzwerk Knotenkapazit�ten. Das hei�t jeder Knoten $v$ hat ein Limit $l(v)$, das regelt wieviel Fluss durch $v$ flie�en darf. Zeige, wie man ein Netzwerk an $G=(V,E)$ mit Knotenkapazit�ten in ein Netzwerk $G'=(V',E')$ ohne Knotenkapazit�ten umwandelt, sodass ein maximaler Fluss in $G'$ den gleichen Wert wie in $G$ hat. Wieviele Knoten hat $G'$?


\bigskip

{\bf Aufgabe 3.} (Anwendung)\\

Wir bearbeiten \textbf{MaxFlow.ipynb} auf Moodle via \href{https://jupyterhub.wolke.uni-greifswald.de}{Jupyter Uni Greifswald}.

\medskip

\bigskip

\end{document}
