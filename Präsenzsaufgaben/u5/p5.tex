\documentclass[12pt,german]{article}
\usepackage[latin1]{inputenc}
\usepackage{textcomp}
\usepackage{epsf}
\usepackage{german}
\usepackage{float}
\usepackage{hyperref}
\usepackage{braket}
\usepackage{amsmath,amsfonts,amssymb,amsxtra}  % z.B. f�r \text im math-modus
\usepackage{pstricks,pst-node,pst-text,pst-3d}
\usepackage{graphicx}
\pagestyle{empty}
\usepackage{tikz}
\evensidemargin0cm
\oddsidemargin0cm
\textwidth16cm
\textheight24cm
\topmargin-2cm
\parindent 0pt
\newtheorem{aufgabe}{Aufgabe}

\begin{document}
\begin{flushright} Becker \end{flushright}
\vspace{-5mm}
\rule{\linewidth}{0.6mm}
\begin{center}{ \Large Datenstrukturen und Effiziente Algorithmen}\\
{\large Wintersemester 2020/21}\vspace{-3mm}
\end{center}
\rule{\linewidth}{0.6mm}
\begin{flushleft}
{\bf \large Pr�senzaufgaben 5}
\end{flushleft}

\vspace*{3mm}

\bigskip
{\bf Aufgabe 1.} (Qual der Wahl)\\
Welche Datenstrukturen bieten sich im Folgenden an, wenn wir effizient Suchen, Einf�gen und L�schen wollen? Wie w�hle ich die IDs (keys)?
\begin{enumerate}
\item Eine Anwendung (z.B. ein Spiel) erstellt dynamisch (d.h. bei Programmaufruf steht die Endanzahl der Objekte noch nicht fest) eine Reihe von Objekten und weist diesen eindeutige IDs zu. Der Einfachheit halber nehmen wir an, dass ein Objekt so lange bestehen bleibt, wie auch die Anwendung, dh. IDs werden nicht invalide.
\item Vorheriger Punkt, aber IDs k�nnen auch invalide werden, wenn Objekte nicht mehr ben�tigt und zerst�rt werden. Was �ndert das wom�gich?
\item Eine Anwendung ben�tigt gro�e Dateien (z.B. Bilder) an verschiedenen Stellen. Es kann vorkommen, dass eine Datei mehrfach angefordert wird. In diesem Fall soll sie nicht mehrfach geladen werden. Man k�nnte bei Anfrage pr�fen, ob sie zuvor schon geladen wurde. (Dateipfade sind Strings)
\item Angenommen man m�chte in einem sehr langen String (z.B. DNA Sequenz) f�r eine Menge von Teilstrings, alle Vorkommen dieser Teilstrings finden und merken. Sp�ter m�chte man f�r einen gegeben Teilstring in konstanter erwarteter Zeit die Anzahl der Vorkommen abfragen (oder z.B. einen Zeiger auf den Kopf einer Liste aller Vorkommen).
\end{enumerate}


\bigskip
{\bf Aufgabe 2.} (Gr��enwahn)\\
Wir betrachten ein \textit{riesiges} Datenfeld, das wir zur direkten Adressierung verwenden wollen. Initialisierung aller Eintr�ge dauert uns zu lange, darum suchen wir eine M�glichkeit, zuf�lligen Datenm�ll von einem tats�chlich gemachten Eintrag zu unterscheiden (idealerweise wollten wir alles mit NULL oder -1 initialisieren, um freie Felder anzuzeigen. Wenn wir das nicht machen, steht in jedem Feld ein zuf�lliger Zeiger). Beschreibe, wie man eine Datenstruktur erh�lt, in der die Operationen \texttt{Search}, \texttt{Insert} und \texttt{Delete} in Zeit $O(1)$ m�glich sind. Jedes gespeicherte Objekt sollte nur $O(1)$ Speicher ben�tigen.\\

\medskip
\textit{Hinweis:} Benutze ein Hilfsarray, das als eine Art Stapel arbeitet und an dem man erkennt, ob ein Slot im riesigen Datenfeld belegt ist oder nicht. Seine Gr��e soll der Anzahl bereits eingef�gter Objekte entsprechen.
\medskip



\bigskip
{\bf Aufgabe 3.} (Erwartete Anzahl an Kollisionen)\\
Sei $h$ eine Hashfunktion, die $n$ Schl�ssel in ein Array der L�nge $m$ hasht. Es gelte die simpe uniform hashing assumption. Was ist die erwartete Anzahl an Kollisionen, d.h. die erwartete M�chtigkeit von $\set{\set{k,l} : k \neq l \land h(k)=h(l)}$?

\bigskip
{\bf Aufgabe 4.} (�berlauf)\\
Angenommen wir speichern $n$ Schl�ssel in einer Hashtabelle der Gr��e $m$ mit Kollisionsaufl�sung durch Verkettung. Die Schl�ssel stammen aus einer Menge $U$ mit $|U| > nm$. Dann existiert eine Teilmenge von $U$ der Gr��e $n$, s.d. alle Elemente auf denselben Slot hashen. (d.h. Zeit f�r Suche ist $\theta(n)$)


\end{document}
