\documentclass[12pt,german]{article}
\usepackage[latin1]{inputenc}
\usepackage{textcomp}
\usepackage{epsf}
\usepackage{german}
\usepackage{float}
\usepackage{hyperref}
\hypersetup{
    colorlinks=true,
    linkcolor=blue,
    filecolor=magenta,      
    urlcolor=cyan,
}
\usepackage{amsmath,amsfonts,amssymb,amsxtra}  % z.B. f�r \text im math-modus
\usepackage{pstricks,pst-node,pst-text,pst-3d}
\usepackage{graphicx}
\pagestyle{empty}
\usepackage{tikz}
\evensidemargin0cm
\oddsidemargin0cm
\textwidth16cm
\textheight24cm
\topmargin-2cm
\parindent 0pt
\newtheorem{aufgabe}{Aufgabe}

\begin{document}
\begin{flushright} Becker \end{flushright}
\vspace{-5mm}
\rule{\linewidth}{0.6mm}
\begin{center}{ \Large Datenstrukturen und Effiziente Algorithmen}\\
{\large Wintersemester 2020}\vspace{-3mm}
\end{center}
\rule{\linewidth}{0.6mm}
\begin{flushleft}
{\bf \large Pr�senzaufgaben 3}
\end{flushleft}

\vspace*{3mm}

{\bf Aufgabe 1.} (Zyklische Rotationen)\\

Nutze den Z-Algorithmus, um folgendes Problem in Linearzeit zu l�sen: Seien $\alpha$, $\beta$ Strings mit $|\alpha|=|\beta|=n$. Ist $\alpha$ eine zyklische Rotation von $\beta$? D.h., besteht $\alpha$ aus einem Suffix von $\beta$ gefolgt von einem Pr�fix von $\beta$? 

\medskip

\textit{Beispiel: $\alpha=defabc$ und $\beta=abcdef$}

\bigskip

{\bf Aufgabe 2.} (Z-Werte am Start)\\

Wenn der Z-Algorithmus $Z_2=q > 0$ findet, k�nnen $Z_3, \dots, Z_{q+2}$ sofort ermittelt werden, ohne zus�tzliche Zeichenvergleiche zu machen. Vertiefe die Details dieser Aussage und finde eine Begr�ndung.


\bigskip

\bigskip

\end{document}
